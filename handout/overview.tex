
We have been talking about TCP, the default transport protocol on the Internet. TCP serves many purposes: it provides reliable, in-order delivery of bytes, it makes sure the sender does not send too fast and overwhelm the receiver (flow control), and it makes sure the sender does not send too fast and overwhelm the network (congestion control). It also aims to be fair: when multiple senders share the same link, they should receive roughly the same proportion of bandwidth.

In class, we discussed TCP Reno. A variant of TCP Reno, NewReno, used to be the standard TCP on the Internet. In the first two checkpoints of this project, you will focus on implementing a transport algorithm that is very similar to TCP Reno. 


However, Reno and NewReno are just two of many congestion control algorithms (CCAs). Companies use different TCPs depending on the context, for example, one TCP for data centers and another for serving web content on the Internet.
For web content, the most common algorithm -- and the default in Linux servers -- is called Cubic. Akamai, the largest content distribution network in the world, uses a proprietary TCP called FastTCP. In the final checkpoint, you will get creative and design your own, new congestion control algorithm for long file transfers on the Internet.

\vspace{5pt}

\noindent With this project, you will gain experience in:
\begin{packed_itemize}
  \item ... building more programs in C, and writing more programs which are compatible with standards for inter-operability.
  \item ... reasoning about designing end-to-end systems when the underlying network is fundamentally unreliable and disorderly.
  \item ... analyzing a program for performance and fairness, designing ways to improve it, and testing those improvements. 
\end{packed_itemize}

\noindent To guide your development process, we have divided the project into three checkpoints.

